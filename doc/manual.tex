\documentclass[11pt]{article}

\usepackage{titling}
\usepackage{url}
\usepackage{enumerate}
\usepackage{pifont}
\usepackage{longtable}
\usepackage[hidelinks]{hyperref}
\usepackage{graphicx}
\usepackage{caption}
\usepackage{subcaption}
\usepackage{tikz}
\usetikzlibrary{arrows,shapes}
\usepackage{pifont}

\setlength{\droptitle}{-3.5cm}

\begin{document}

\title{{\bf Axe: a consistency checker for memory-subsystem traces}\\
{\large Version 1.3, April 2016}}
\author{Matthew Naylor and Simon Moore, \\
University of Cambridge Computer Laboratory}
\date{}
\maketitle

\tableofcontents

\newpage

\setlength{\parskip}{1em}
\renewcommand{\thefootnote}{\fnsymbol{footnote}}

\newcommand{\cmark}{\ding{51}}
\newcommand{\xmark}{\ding{55}}

\section{Introduction}

Axe is a tool that aids automatic black-box testing of the
memory subsystems found in modern multi-core processors.  Given a
trace containing a set of top-level memory requests and responses, Axe
determines if the trace is valid according to a range of \emph{memory
consistency models}.  It is designed to be used as the oracle in an
automated hardware test framework, quickly checking large memory
traces that result from mechanically-generated sequences of memory
operations \cite{BlueCheck}.
%It can also assist bug diagnosis by enabling
%testing strategies that search
%for small failing cases \cite{BlueCheck}.
Despite
the large amount of non-determinism present in memory consistency
models, Axe can handle large traces involving many cores.

Axe supports a spectrum of five consistency models listed in Table
\ref{Table:Models}, each one permitting a subset of the behaviours
allowed by the next.
We validate Axe by testing equivalence against simpler, non-optimised
trace-checkers for the same models.
Axe is available from \url{http://www.github.com/CTSRD-CHERI/axe}.
The name ``Axe'' comes from the use of axiomatic rules to decide the
validity of traces.

Example applications of Axe to open-source memory-subsystem
implementations are available in Appendix A.

\begin{table}
\renewcommand{\arraystretch}{1.5}
\begin{center}
\begin{tabular}{lll}
          & \textbf{Model}  & \textbf{Reference} \\
          & SC     & Sequential Consistency \cite{SC} \\
$\subset$ & TSO    & Total Store Order \cite{SPARC} \\
$\subset$ & PSO    & Partial Store Order \cite{SPARC} \\
$\subset$ & WMO\footnotemark[2]
                   & Weak Memory Order \cite{SPARC} \\
$\subset$ & POW    & POWER model \cite{POWER} \\
\end{tabular}
\end{center}
\caption{Total order of supported memory consistency models.}
\label{Table:Models}
\end{table}

\footnotetext[2]{WMO is equivalent to SPARC RMO \cite{SPARC} except
that it forbids reordering of loads to the same address, making it a
subset of modern relaxed models such as POWER \cite{POWER}.}

\subsection{Problem definition}
\label{Section:ProbDef}

Given a trace containing a set of memory requests and responses
(including loads, stores, atomic read-modify-writes, memory barriers,
and optional timestamps) initiated by concurrent processor cores (or
``hardware threads''), Axe determines if the trace satisfies one of
the consistency models listed in Table \ref{Table:Models}.  We define
the memory trace format in \S\ref{Section:TraceFormat} and present
the semantics of each consistency model in
\S\ref{Section:SPARCModels} and \S\ref{Section:POWERModel}.

Following Gibbons \cite{Gibbons} and Manovit \cite{Manovit}, we assume
that the address-value pair of every store in a trace is unique, i.e.
the same value is never written to the same address twice.  This
reduces the amount of non-determinism in a model as it allows the
store read by any load to be uniquely identified. The restriction is
easily met by an automatic test generator and is justified by the
fact that the actual values being stored do not typically affect any
interesting hardware behaviour.  But it does mean that our tool cannot
be used for checking memory traces that arise during execution of
arbitrary software applications, which are unlikely to meet this
restriction.

Another technique for reducing non-determinism is to modify the
hardware to emit extra trace information such as the order in which
writes reach a particular internal merge point.  However, for now we
treat the memory subsystem as a \emph{black box} and do not inspect or
modify its internals in any way: we would
like our tool to be as easy as possible to use, i.e. not requiring
modifications to the system under test.

\subsection{Background}

Axe is heavily inspired by the work of Manovit et al.
\cite{Manovit,TSOToolISCA,TSOToolSPAA,TSOToolHPCA} and their
\emph{TSOtool} \cite{TSOTool}.  TSOtool generates pseudo-random
multi-threaded programs, runs them on a multiprocessor, and compares
the results against the \emph{Total Store Order} specification (TSO)
to reveal potential discrepancies.  A key contribution of the work is
a state-of-the-art conformance checking algorithm for TSO that can
handle long-running programs -- on the order of millions of memory
operations and hundreds of cores -- \emph{despite} this being an NP-complete
problem.  TSOtool, and variants of it, have been used with success at
Sun Microsystems \cite{Manovit} and Intel \cite{IntelChecker}.
Unfortunately, ``TSOTool is a proprietary program of Sun
Microsystems'' \cite{TSOTool}, and only supports the TSO model.  Our
checking algorithm for the SPARC models (SC, TSO, PSO, WMO) is a
generalisation of TSOtool's algorithm; it has a worse time and space
complexity but still performs very well in practice.

Our checking algorithm for the POWER model (POW) is inspired by the
operational semantics of Sarkar et al. \cite{POWER} and the
associated PPCMEM tool \cite{PPCMEM}, but offers performance
sufficient for specification-based testing purposes.

\section{Trace format}
\label{Section:TraceFormat}

We introduce the syntax of memory-subsystem traces by way of example.

\paragraph{Example 1} Here is a simple trace consisting of five
operations running on two threads.  

\begin{verbatim}
0: M[1] := 1
0: sync
0: M[0] == 0
1: M[0] := 1
1: M[1] == 0
\end{verbatim}

\noindent The number before the \verb#:# denotes the thread id.  The
first line can be read as: thread \verb#0# stores value \verb#1# to
memory location \verb#1#.  The second line as: thread \verb#0#
performs a memory barrier.  And the final line as: thread \verb#1#
reads value \verb#0# from memory location \verb#1#.

The initial value of every memory location is implicitly \verb#0#.
For any read of a value other than \verb#0#, there must exist a write
of that value to the same address in the trace, otherwise the trace is
said to be malformed.  As mentioned in \S\ref{Section:ProbDef}, we
also require that the address-value pair of every write is unique.

The textual order of operations with the same thread id is the order
in which those operations were issued to the memory subsystem by
that thread.  We refer to this order as the \emph{thread-order}.
No ordering is implied by the textual order of
operations with different thread ids.  In the above example, the write
by thread \verb#1# is not ordered in any way with respect to any of
the operations by thread \verb#0#, but it is thread-order-before the
read by thread \verb#1#.

\paragraph{Example 2} Here is another trace, illustrating timestamps.

\begin{verbatim}
0: M[0] := 1
0: sync
0: M[1] := 1
1: M[1] == 1    @ 100 : 110
1: M[0] == 0    @ 115 :
\end{verbatim}

\noindent The operation on the fourth line contains a begin-time of
\verb#100# and an end-time of \verb#110#, denoting the times at which
the request was submitted and the response received respectively.
Operations that perhaps do not generate a response, such as a store,
can simply leave the end-time unspecified\footnote{Axe
currently disallows end-times for store operations, making it clear
that this information is not used.}.
In fact, all timestamps are completely optional, for a few reasons:

\begin{itemize}

\item some consistency models are unaffected by timestamps;

\item timestamps may not be available, depending on how the
traces are produced; and

\item example traces are easier to read if only the interesting or
relevant timestamp information is supplied.

\end{itemize}

In some consistency models however, timestamps can affect whether or
not a trace is allowed.  In the above example, the timestamps indicate
that first load by thread \verb#1# must have finished before the
second load by thread \verb#1# begins, implying that the memory
subsystem could not have executed the operations out-of-order.  In the
SPARC and POWER architectures, a programmer can arrange such a
dependency by having the address of the second load be dependent on
the result of the first load -- a so-called \emph{address dependency}
\cite{POWER}.  Other kinds of dependency include \emph{data
dependencies} (where the value of a store is dependent on the result
of a preceding load) and \emph{control dependencies} (where an
operation is control-flow dependent on the result of preceding load).
These program-level dependencies become observable in the memory
trace as end-time-before-begin-time dependencies.

By default, we consider timestamps to be \emph{local to each thread},
i.e. we do not use timestamps to infer ordering between operations
that run on different threads.  This means we can test hardware in
which the threads are running in separate clock domains, for example.
However, if the \verb!-g! command line flag is specified then Axe can
assume a global clock, and compare timestamps of operations running on
different threads.  Currently, we only exploit the \verb!-g! flag in
our POW model (\S\ref{Section:POWERModel});

There is no explicit support in Axe for \emph{cancelled operations}
which often arise in modern CPUs due to speculative execution or
exceptions.  Traces containing such operations can still be checked
using Axe by simply replacing them with no-ops.  There is also no
support for \emph{mixed-width} accesses at present:  Axe abstracts
over the width of each memory location and hence the width may vary
between traces, but \emph{not within a trace}.

\paragraph{Example 3} Here is third trace, this time containing
three operations, the first of which is an atomic read-modify-write
operation.

\begin{verbatim}
0: <M[0] == 0; M[0] := 1>
1: M[0] := 2
1: M[0] == 1
\end{verbatim}

\noindent The first line can be read as: thread \verb#0#
\emph{atomically} reads value \verb#0# from memory location \verb#0#
and updates it to value \verb#1#.  The two memory addresses in an
atomic operation must be the same, otherwise the trace is malformed.
A common way of expressing atomic operations in RISC instruction sets
is via a pair of \emph{load-linked} and \emph{store-conditional}
operations.  At the trace level, it is straightforward to convert such
a pair of operations into a single read-modify-write operation:

\begin{itemize}

\item if the store-conditional fails, then remove it from the trace
and convert the load-linked to a standard load;

\item otherwise, convert both operations to a single read-modify-write
operation.

\end{itemize}

\noindent For read-modify-write operations, the end-time simply
denotes the time at which the read-response is received.  Currently,
Axe has no concept of an end-time on a write operation.

\paragraph{Example 4} The following trace illustrates \verb#final#
constraints.

\begin{verbatim}
0: M[0] := 1
0: M[1] := 1
1: M[1] := 2
1: M[0] == 0
final M[1] == 2
\end{verbatim}

\noindent The \verb#final# line states that final value at location
\verb#1# viewed by all threads after all operations have completed is
\verb#2#.  These \verb#final# constraints are entirely optional and
are primarily supported so that litmus tests (used for testing Axe)
can be neatly expressed as traces.

\section{Command-line usage}
\label{Section:CmdLine}

Axe can be invoked as follows:
\begin{verbatim}
  axe check <MODEL> <FILE> [-g] [-i]
\end{verbatim}
\noindent where \verb!<MODEL>! is \verb!SC!, \verb!TSO!, \verb!PSO!,
\verb!WMO!, or \verb!POW!; \verb!<FILE>! is the name of a file
containing a trace or ``\verb!-!'' to read a
trace from standard input.  The optional \verb!-g! flag (currently
only used in the \verb!POW! model) indicates that a global clock domain
may be assumed (see \S\ref{Section:TraceFormat} for more details).
The optional \verb!-i! flag indicates that timestamps in the trace
should be ignored.

The output is either ``\verb!OK!'', denoting that the trace is allowed by
the specified model, or ``\verb!NO!'' if it is forbidden.  If the trace is
malformed or does not meet the necessary constraints, an error message
will be reported.
To illustrate, if the file \verb!trace.axe! contains:
\begin{verbatim}
  0: M[1] := 1
  0: M[0] == 0
  1: M[0] := 1
  1: M[1] == 0
\end{verbatim}
\noindent then the command
\begin{verbatim}
  axe check TSO trace.axe
\end{verbatim}
\noindent will output ``\verb!OK!''.

\subsection*{Multiple traces per file}

Axe allows a single file to contain multiple traces, with each trace
terminated by a line containing the text ``\verb!check!''.  It also allow
comments (lines beginning with the character ``\verb!#!'') in trace
files.
To illustrate, if the file \verb!traces.axe! contains:
\begin{verbatim}
  # Trace 1
  0: M[1] := 1
  0: M[0] == 0
  1: M[0] := 1
  1: M[1] == 0
  check

  # Trace 2
  0: M[0] := 1
  0: M[1] := 1
  1: M[1] == 1
  1: M[0] == 0
  check
\end{verbatim}
\noindent then the command
\begin{verbatim}
  axe check TSO traces.txt
\end{verbatim}
\noindent will output:
\begin{verbatim}
  OK
  NO
\end{verbatim}
\noindent That is, one decision per trace, in order.

\subsection*{Interaction}

It is straightforward to connect Axe to other tools such as HDL
simulators: any program can simply \verb!popen()! Axe, specifying the
input file as ``\verb!-!'', and communicate with it via pipes.

\subsection*{Testing}

Axe also supports the invocation pattern:
\begin{verbatim}
  axe test <MODEL> <FILE> <FILE> [-g] [-i]
\end{verbatim}
\noindent where the arguments are the same as before, except for the
introduction of the
second \verb!<FILE>! argument which specifies a file of expected
outcomes (i.e. ``\verb!OK!'' or ``\verb!NO!''), one for each trace in the trace
file.  Axe reports an error if any trace does not give
the expected outcome.  The purpose of this mode is to support testing of
Axe itself.  There are a large number of tests and expected outcomes
in the ``\verb!tests!'' subdirectory of the Axe distribution.

\subsection*{Shrinking traces}

When a trace fails a given consistency model, Axe
simply reports back ``\verb!NO!''.  This is not particularly helpful
in understanding \emph{why} a trace violates a model,
especially when the trace is
long.  In such cases, the \verb!axe-shrink.py! script
(included in the \verb!src! subdirectory) can be used to search for a
small subset of the trace that fails the model.
To illustrate, suppose the file \verb!failure.axe! contains a 260-line
trace that fails the TSO model.  Running
\begin{verbatim}
  axe-shrink.py TSO failure.axe
\end{verbatim}
might give:
\begin{verbatim}
  Pass 0
  Omitted 241 of 260         
  Pass 1
  Omitted 256 of 260         
  Pass 2
  Omitted 256 of 260         
  0: M[2] := 46 @ 497:
  1: M[2] == 46 @ 280:513
  1: M[2] := 61 @ 729:
  1: M[2] == 46 @ 854:979
\end{verbatim}

\noindent The 260-line trace has been reduced to 4 lines.  This simple
shrinker tries, in reverse trace order, to drop each operation in turn
and this is repeated until a fixed-point is reached.

For large traces, we have developed \verb!axe-big-shrink.py!, which
works by applying each of the following rewrite rules for $retry$
attempts before moving on to the next rule.  Each rule is conditioned
on the resulting trace still violating the model.

\begin{enumerate}
\item pick an address and drop all accesses to that address;
\item drop a random subset of $n$ loads;
\item drop a random subset of $n$ stores which write
a value that is never read;
\item same as (3) except for read-modify-write operations.
\end{enumerate}

\noindent After that, it resorts back to the simple shrinking
algorithm.  For suitable choices of $retry$ and $n$, it is both
effective and fast.

\section{SPARC models}
\label{Section:SPARCModels}

This section presents an operational semantics for the SC, TSO, PSO
and WMO models supported by Axe.  The common feature of these models
is the existence (or illusion) of a \emph{single shared memory}: if a
write by one thread is observed by another then it must be observable
to all threads.  Sometimes known as \emph{multi-copy atomicity} or
\emph{global store atomicity}, this property is typically provided by
hardware that implements a single-writer cache coherence protocol.

We define the behaviours allowed by each model using an abstract
machine consisting of a state and a set of state-transition rules.  In
each case, the state consists of:

\begin{itemize}

\item A trace $T$ (a sequence of operations in the format given in
\S\ref{Section:TraceFormat}).

\item A mapping $M$ from memory addresses to values.

\item A mapping $B$ from thread ids to sequences of operations.  We
call $B(t)$ the \emph{local buffer} of thread $t$.

\end{itemize}

In the \emph{initial state}, $T$ is the trace we wish to check, $M(a)
= 0$ for each address $a$, and $B(t) = []$ for each thread $t$.
(Notation: $[]$ denotes the empty sequence.)

Using the state-transition rules, if there is a path from the initial
state to a state in which $T = []$ and $B(t) = []$ for all threads
$t$, where $M$ satisfies all the \verb#final# constraints in this
final state, then we say that the machine accepts the initial trace
and that the trace $T$ is allowed by the model.  Otherwise, it is
disallowed by the model.

\subsection{Sequential Consistency (SC)}

SC has just one state-transition rule.

\paragraph{Rule 1}

Pick a thread $t$ non-deterministically.  Remove the first operation
$i$ executed by $t$ from the trace.

\begin{enumerate}
\item
     If $i = \texttt{M[}a\texttt{]:=}~v$ then update $M(a)$ to $v$.

\item
     If $i = \texttt{M[}a\texttt{]==}~v$ and $M(a) \neq v$
     then $\textbf{fail}$.

\item
     If $i = \texttt{<M[}a\texttt{]==}~v_0\texttt{; M[}a\texttt{]:=}~v_1
     \texttt{>}$ then:

\begin{enumerate}[i]
\item
          if $M(a) \neq v_0$ then $\textbf{fail}$;
\item
          else: update $M(a)$ to $v_1$.
\end{enumerate}
\end{enumerate}

\noindent We use the term $\textbf{fail}$ to denote that the
transition rule \emph{cannot} be applied under the chosen values for
the non-deterministic variables.  In this case $t$ is the only
non-deterministic variable.  

\subsection{Total Store Order (TSO)}

In TSO, each thread has a local store buffer.  Before presenting the
semantics, we give a few examples of TSO behaviour.

\paragraph{Example (SB)} Here is a sample trace that is allowed by
TSO but forbidden by SC.

\begin{verbatim}
0: M[1] := 1
0: M[0] == 0
1: M[0] := 1
1: M[1] == 0
\end{verbatim}

\noindent There are six possible interleavings of each thread's
operations but none result in both reads returning zero.  However,
under TSO \cite{SPARC}, stores may be buffered locally by a thread,
allowing subsequent loads to complete before the buffered stores can
be observed by other threads.

\paragraph{Example (SB+syncs)}  Under TSO, the above behaviour can be
prevented by inserting \verb!sync! operations that cause the local
buffers to be flushed.  The following trace is forbidden.

\begin{verbatim}
0: M[1] := 1
0: sync
0: M[0] == 0
1: M[0] := 1
1: sync
1: M[1] == 0
\end{verbatim}

\noindent In general, placing a \verb!sync! between every pair of
thread-ordered operations restores SC behaviour.

\paragraph{Example (SB+RMWs)} Another way to prevent the
relaxed behaviour in the SB example is to replace each write with an
atomic read-modify-write.  The following trace is forbidden.

\begin{verbatim}
0: { M[1] == 0; M[1] := 1 }
0: M[0] == 0
1: { M[0] == 0; M[0] := 1 }
1: M[1] == 0
\end{verbatim}

\noindent Under TSO, read-modify-write has the side-effect of
flushing the store buffer.

\subsubsection*{Operational Semantics}

We define TSO using two rules.  The first is similar to Rule 1 of SC,
modified to deal with writing to and reading from the store buffers.
The second deals with evicting elements from the buffers to memory.

\paragraph{Rule 1}

Pick a thread $t$ non-deterministically.  Remove the first operation
$i$ executed by $t$ from the trace.

\begin{enumerate}
\item 
     If $i = \texttt{M[}a\texttt{]:=}~v$ then append $i$ to $B(t)$.

\item 
     If $i = \texttt{M[}a\texttt{]==}~v$ then let $j$ be the latest
     operation of the form $\texttt{M[}a\texttt{]:=}~w$ in
     $B(t)$ and:
     
       i.   if $j$ exists and $v \neq w$ then $\textbf{fail}$.

       ii.  if $j$ does not exist and $M(a) \neq v$ then $\textbf{fail}$;

\item 
     If $i = \texttt{sync}$ and $B(t) \neq []$ then $\textbf{fail}$.

\item
     If $i = \texttt{<M[}a\texttt{]==}~v_0\texttt{; M[}a\texttt{]:=}~v_1
     \texttt{>}$ then:

\begin{enumerate}[i]
\item
           if $B(t) \neq []$ then $\textbf{fail}$;

\item
           else if $M(a) \neq v_0$ then $\textbf{fail}$;

\item
           else: update $M(a)$ to $v_1$.
\end{enumerate}
\end{enumerate}

\paragraph{Rule 2}

Pick a thread $t$ non-deterministically. Remove the first operation
$\texttt{M[}a\texttt{]:=}~v$ from $B(t)$ and update $M(a)$ to v.

\subsection{Partial Store Order (PSO)}

PSO is similar to TSO but relaxes the order in which writes can be
evicted from the buffer.  In particular: writes to different addresses
can be evicted out-of-order.

\paragraph{Example (MP)} The following trace is allowed by PSO but
forbidden by TSO.

\begin{verbatim}
0: M[0] := 1
0: M[1] := 1
1: M[1] == 1
1: M[0] == 0
\end{verbatim}

\noindent The writes may be evicted \emph{out-of-order} from the local
buffer on thread 0, allowing thread 1 to see the second write before
it sees the first.

\paragraph{Example (MP+sync+po)} The above relaxed behaviour can be
prevented by inserting a \verb!sync! between the writes.  The
following trace is forbidden by PSO.

\begin{verbatim}
0: M[0] := 1
0: sync
0: M[1] := 1
1: M[1] == 1
1: M[0] == 0
\end{verbatim}

\paragraph{Example (MP+RMW)}  Under TSO, atomic operations have the
side-effect of flushing the local write buffer.  Under PSO, only
writes to the same address are flushed, hence the following trace
is allowed under PSO.

\begin{verbatim}
0: M[0] := 1
0: { M[1] == 0; M[1] := 1 }
1: M[1] == 1
1: M[0] == 0
\end{verbatim}

\subsubsection*{Operational Semantics}

\paragraph{Rule 1}

This is identical to Rule 1 of TSO except that clause 4 becomes:

\begin{enumerate}
\setcounter{enumi}{3}
\item
     If $i = \texttt{<M[}a\texttt{]==}~v_0\texttt{; M[}a\texttt{]:=}~v_1
     \texttt{>}$ then:
\begin{enumerate}[i]

\item
           if any operation in $B(t)$ refers to address $a$ then
           $\textbf{fail}$;

\item
           else if $M(a) \neq v_0$ then $\textbf{fail}$;

\item
           else: update $M(a)$ to $v_1$.
\end{enumerate}
\end{enumerate}

\paragraph{Rule 2}

Non-deterministically pick a thread $t$ and an address $a$.  Remove
the first operation that refers to address $a$,
$\texttt{M[}a\texttt{]:=}~v$, from $B(t)$ and update $M(a)$ to $v$.

\subsection{Weak Memory Order (WMO)}

WMO is a relaxation of PSO in which load operations, like stores,
become non-blocking.  Unlike SPARC's RMO, it forbids reordering of
loads to the same address, making it a subset of modern relaxed models
such as POWER.  In all other respects, it is equivalent to RMO.

\paragraph{Example (MP+sync+po resisted)} This example, forbidden by
PSO, is allowed by WMO because while the writes must occur in
order, the loads (to different addresses) may happen out-of-order.

\paragraph{Example (MP+syncs)}  If a sync is also placed between the
two loads as follows, then the relaxed behaviour becomes forbidden.

\begin{verbatim}
0: M[0] := 1
0: sync
0: M[1] := 1
1: M[1] == 1
1: sync
1: M[0] == 0
\end{verbatim}

\paragraph{Example (MP+sync+dep)} Alternatively, a dependency between
the two loads, in the form of a ``begin-time after an end-time'' may
also be used to keep the loads in order.  The following trace is
forbidden by WMO.

\begin{verbatim}
0: M[0] := 1
0: sync
0: M[1] := 1
1: M[1] == 1   @ 100:110
1: M[0] == 0   @ 115:
\end{verbatim}

\noindent The fact that the second load begins after the first one
completes is enough, under WMO, to imply that the memory subsystem
cannot reorder them.

\paragraph{Example (LB)} The MP+sync example demonstrates reordering
of two loads, but WMO also allows reordering of a load followed by a
store (when the addresses are different).  The following trace is
allowed by WMO.

\begin{verbatim}
0: M[0] == 1
0: M[1] := 1
1: M[1] == 1
1: M[0] := 1
\end{verbatim}

\paragraph{Example (LB+syncs \& LB+addrs)} As expected, a \verb!sync!
after each load will prevent the LB behaviour.  So too will a
timestamp dependency between each load and store.

\subsubsection*{Operational Semantics}

\paragraph{Rule 1} 

Pick a thread $t$ non-deterministically.  Remove the first operation
$i$ executed by $t$ from the trace.

\begin{enumerate}
\item
     If $i = \texttt{sync}$ and $B(t) = []$ then succeed;
\item
     Otherwise: $\textbf{fail}$.
\end{enumerate}

\paragraph{Rule 2}

Non-deterministically pick a thread $t$ and an address $a$.  From the
trace, remove the first operation $i$ on thread $t$ that satisfies the
condition: (a) $i = \texttt{sync}$; or (b) $i$ accesses address $a$
and no operation that precedes $i$ in thread-order has an end-time
that precedes the begin-time of $i$.

\begin{enumerate}
\item 
     If $i = \texttt{sync}$ then $\textbf{fail}$.

\item 
     If $i = \texttt{M[}a\texttt{]:=}~v$ then append $i$ to $B(t)$.

\item 
     If $i = \texttt{M[}a\texttt{]==}~v$ then let $j$ be the latest
     operation of the form $\texttt{M[}a\texttt{]:=}~w$ in
     $B(t)$ and:
     
       i.   if $j$ exists and $v \neq w$ then $\textbf{fail}$.

       ii.  if $j$ does not exist and $M(a) \neq v$ then $\textbf{fail}$;

\item
     If $i = \texttt{<M[}a\texttt{]==}~v_0\texttt{; M[}a\texttt{]:=}~v_1
     \texttt{>}$ then:

\begin{enumerate}[i]
\item
           if $B(t) \neq []$ then $\textbf{fail}$;

\item
           else if $M(a) \neq v_0$ then $\textbf{fail}$;

\item
           else: update $M(a)$ to $v_1$.
\end{enumerate}
\end{enumerate}

\paragraph{Rule 3}

Non-deterministically pick a thread $t$ and an address $a$.  Remove
the first operation that refers to address $a$,
$\texttt{M[}a\texttt{]:=}~v$, from $B(t)$ and update $M(a)$ to $v$.

\subsection{Axiomatic definitions}

This section presents the axiomatic definitions of the SC, TSO, PSO and
WMO models upon which the Axe checking algorithm for these
models is based.  In this section, we consider a read-modify-write
operation to be both a ``load'' and a ``store''.

To begin, it is helpful to distinguish between two different orderings
over operations in the trace:

\begin{itemize}

\item
\emph{Thread Order}: for any given thread, the textual order of
operations in the trace issued by that thread.

\item
\emph{Memory Order}: a total order over all operations in the trace.

\end{itemize}

All valid traces under these models must satisfy the following
property (\textbf{value axiom}): the value returned by a load from
address $a$ equals the value of the latest store (in memory
order) from the set $Local \cup Global$ where $Local$ is the set of
stores to address $a$ that precede the load in \emph{thread order}
and $Global$ is the set of stores to address $a$ that precede the load
in \emph{memory order}.

Depending on the model, the following \textbf{local axioms} on
operations $i$ and $j$ from the same thread must also be satisfied.

\subsubsection*{Sequential Consistency (SC)}

If $i$ precedes $j$ in thread-order then $i$ must precede
$j$ in memory order.

\subsubsection*{Total Store Order (TSO)}

If $i$ precedes $j$ in thread-order then $i$ must precede
$j$ in memory order \textbf{when}:

\begin{itemize}
\item $i$ is a load; or
\item $i$ and $j$ are stores; or
\item $i$ is a \verb!sync! or $j$ is a \verb!sync!.
\end{itemize}

\subsubsection*{Partial Store Order (PSO)}

If $i$ precedes $j$ in thread-order then $i$ must precede
$j$ in memory order \textbf{when}:

\begin{itemize}
\item $i$ is a load; or
\item $i$ and $j$ are stores \emph{to the same address}; or
\item $i$ is a \verb!sync! or $j$ is a \verb!sync!.
\end{itemize}

\subsubsection*{Weak Memory Order (WMO)}

If $i$ precedes $j$ in thread-order then $i$ must precede
$j$ in memory order \textbf{when}:

\begin{itemize}

\item $i$ is a load and $j$ accesses the same address; or

\item $i$ and $j$ are stores to the same address; or

\item $i$ is a \verb!sync! or $j$ is a \verb!sync!; or

\item $i$ is a load with end-time $t_0$ and $j$ has begin-time
$t_1$ and $t_0 < t_1$.

\end{itemize}

\subsection{Checking algorithm}

In this section, we generalise an algorithm by Manovit \cite{Manovit}
for checking traces against the TSO model to support the SC, TSO, PSO
\emph{and} WMO models.  The central data structure used by this
algorithm is the \emph{analysis graph} in which each node denotes an
operation from the trace, and each edge denotes that the source node
precedes the destination node in memory order.

%Note that the analysis graph represents a partial order.

\subsubsection*{Simple algorithm}

Starting with an empty analysis graph, a simple checking algorithm is
as follows.

\begin{enumerate}

\item Add each operation in the trace as a node to the analysis graph.

\item Add the edges implied by the local axioms defined
above.

\item Apply the two edge-introduction rules shown in Figure
\ref{Fig:IntroRules} to the graph. 

\item Add an edge from each read $\texttt{M[}x\texttt{] == 0}$ to the
first store $\texttt{M[}x\texttt{] := }a$ on each
thread.  This ensures that any read of zero (initial value) from
address $x$ must happen before any writes to address $x$.

\item Apply a standard topological sort procedure to the analysis
graph with the following tweak: every time a store operation
$\texttt{M[}x\texttt{] := }a$ is removed from the graph, add an edge
from each load $\texttt{M[}x\texttt{] == }a$ to the next
unpicked store $\texttt{M[}x\texttt{] := }b$ on each
thread.  This ensures that any read of the current value at address
$x$ must happen before any store of another value to address $x$.

\item If a topological sort can be found, i.e.  a total order of
operations exists that satisfies the memory order constraints, then
the trace is valid, otherwise it is invalid.

\end{enumerate}


\begin{figure}[p]
\centering
\begin{subfigure}{.5\textwidth}
  \centering
\vspace{7.7mm}
\begin{tikzpicture}
  [scale=.6,auto=left,every node/.style={ellipse,fill=gray!20}]
  \node (n1) at (1,4) {$\texttt{M[}x\texttt{]}~\texttt{:=}~a$};
  \node (n3) at (7,1) {$\texttt{M[}x\texttt{]}~\texttt{==}~a$};

  \draw[dotted,bend left, arrows={-{triangle 45}}] (n1) to (n3);
  \draw[bend right,arrows={-{triangle 45}}] (n1) to (n3);
  \node[fill=none] (x) at (2.63, 1.7) {\rotatebox{-10}{\xmark}};
  \node[fill=none] (po) at (2.2,1.2) {\emph{to}};
\end{tikzpicture}
  \caption{}
  \label{Fig:Intro1}
\end{subfigure}%
\begin{subfigure}{.5\textwidth}
  \centering
\begin{tikzpicture}
  [scale=.6,auto=left,every node/.style={ellipse,fill=gray!20}]
  \node (n1) at (1,4) {$\texttt{M[}x\texttt{]}~\texttt{:=}~a$};
  \node (n2) at (7,4) {$\texttt{M[}x\texttt{]}~\texttt{:=}~b$};
  \node (n3) at (7,1) {$\texttt{M[}x\texttt{]}~\texttt{==}~a$};

  \draw[arrows={-{triangle 45}}] (n2) to (n3);
  \draw[dotted,arrows={-{triangle 45}}] (n2) to [bend right=45] (n1);
  \node[fill=none] (po) at (7.5,2.5) {\emph{to}};
\end{tikzpicture}
  \caption{}
  \label{Fig:Intro2}
\end{subfigure}%

\caption{Edge-introduction rules. In (a) the dotted memory-order
edge is introduced if the solid thread-order edge, labelled
\emph{to}, does not exist.  In (b) the dotted memory-order edge
is introduced if the thread-order edge, labelled \emph{to}, exists.}
\label{Fig:IntroRules}
\end{figure}

\begin{figure}[p]
\centering
\begin{subfigure}{.5\textwidth}
  \centering
\begin{tikzpicture}
  [scale=.6,auto=left,every node/.style={ellipse,fill=gray!20}]
  \node (n1) at (4,4) {$\texttt{M[}x\texttt{]}~\texttt{:=}~a$};
  \node (n2) at (1,1)  {$\texttt{M[}x\texttt{]}~\texttt{:=}~b$};
  \node (n3) at (7,1)  {$\texttt{M[}x\texttt{]}~\texttt{==}~a$};

  \draw[arrows={-{triangle 45}}] (n1) to (n2);
  \draw[arrows={-{triangle 45}}] (n1) to (n3);
  \draw[dotted,arrows={-{triangle 45}}] (n3) to [bend left=45] (n2);
\end{tikzpicture}
  \caption{}
  \label{Fig:Infer1}
\end{subfigure}%
\begin{subfigure}{.5\textwidth}
  \centering
\begin{tikzpicture}
  [scale=.6,auto=left,every node/.style={ellipse,fill=gray!20}]
  \node (n1) at (1,4) {$\texttt{M[}x\texttt{]}~\texttt{:=}~b$};
  \node (n2) at (7,4) {$\texttt{M[}x\texttt{]}~\texttt{:=}~a$};
  \node (n3) at (4,1) {$\texttt{M[}x\texttt{]}~\texttt{==}~b$};

  \draw[arrows={-{triangle 45}}] (n1) to (n3);
  \draw[arrows={-{triangle 45}}] (n2) to (n3);
  \draw[dotted,arrows={-{triangle 45}}] (n2) to [bend right=45] (n1);
\end{tikzpicture}
  \caption{}
  \label{Fig:Infer2}
\end{subfigure}
\caption{Edge-inference rules proposed by
Manovit \cite{Manovit} (our representation).  In
each case, if the solid memory-order edges are known to exist,
either directly or
by transitivity, then the dotted memory-order edge can be inferred.}
\label{Fig:InferRules}
\end{figure}

The key inefficiency of this algorithm is the non-determinism present
in the topological sort.  At any stage, there may exist several store
operations that can be removed next.  If a bad choice is made, the
algorithm must backtrack since an alternative choice may lead to
success.  (The order of stores to each address is not known in
advance.)

\subsubsection*{Reducing non-determinism}

Manovit proposes the two rules shown in Figure \ref{Fig:InferRules} as
a way of inferring new edges in the analysis graph, greatly
reducing the amount of non-determinism in the topological sort.
Notice that applying these rules can introduce edges which enable the
rules to be applied again.  Therefore it is desirable to apply the
rules repeatedly until a fixed-point is reached, i.e. until no new
edges are inferred.

This leads to two modifications of the simple algorithm above: first,
add a new step after step (3) that applies the inference rules until a
fixed-point; second, every time a store is removed from the
graph in step (5), and new edges are added, reapply the inference rules
until a fixed-point is reached.

\subsubsection*{Reducing rule-application sites}

Applying the inference rules at all matching sites in the analysis
graph would be extremely inefficient and, fortunately, unnecessary.
Instead, it is sufficient to apply each rule once for each store $s$
of the form $\texttt{M[}x\texttt{] := } a$ with:

\begin{itemize}

\item for rule \ref{Fig:Infer1}, node $\texttt{M[}x\texttt{] := } b$
bound to the earliest store to address $x$ that succeeds
$s$ in the analysis graph;

\item for rule \ref{Fig:Infer2}, node $\texttt{M[}x\texttt{] == } b$
bound to the earliest load to address $x$ that succeeds
$s$ in the analysis graph.

\end{itemize}

\noindent While there may exist several bindings that satisfy the
above constraints (the earliest successor may not be unique in a
partial order), the number of application sites to consider is greatly
reduced.

\subsubsection*{Determining the earliest successors}

The problem now is: starting from any store operation, how do we
efficiently determine the next load and store to the same address in
the analysis graph?

To answer this, we maintain two data structures.  The first is the
mapping $nextLoad(op, t, a)$ that gives the next \emph{load} (in the
analysis graph) to address $a$ on thread $t$ from operation $op$.
(Since loads to the same address on a given thread are totally ordered
under all models, this mapping is a function, i.e. unambiguous.)
Initially, it is computed by a backward analysis, propagating the next
load for each $(a, t)$ pair backwards along the edges of the analysis
graph, in reverse-topological order.  At a fork point, the information
at several nodes is merged by taking the minimum load in thread order
for each $(a, t)$ pair.  When a new edge $i \rightarrow j$ is added to
the graph, the $nextLoad$ mapping is updated by applying the same
propagation method backwards from node $j$ until no new updates to the
mapping are made.

The second data structure we maintain is the mapping $nextStore$,
identical to $nextLoad$ but giving the next store instead of the next
load.  These two data structures have a number of uses:

\begin{itemize}

\item the inference rules from Figure \ref{Fig:InferRules} can be
efficiently applied;

\item the existence of a path from a store to any load or store can
be determined in constant-time, avoiding the addition of
redundant edges to the graph;

\item similarly, we can be determine in constant-time whether or not
the addition of an edge to the graph will lead to a cycle, allowing
immediate failure detection;

\item when adding an edge to the graph, the backwards propagation
method used to update the $nextLoad$ and $nextStore$ mappings will
naturally visit all the nodes at which the inference rules must be
re-applied.

\end{itemize}

\subsubsection*{Comparison to Manovit's algorithm}

When specialising the algorithm to the TSO model, it is possible to
simplify the $nextLoad$ and $nextStore$ mappings.  Instead of mapping
each $(op, a, t)$ triple to the next load or next store, it is
sufficient to map each $(op, t)$ pair.  This is because all loads by
the same thread are totally ordered under TSO, and so too are all
stores by the same thread.  Once the next load on some
thread is determined, the next load to a particular address on that
thread can be easily found by looking at the static thread order.
Consequently, the size of these data structures reduces from $2 \times
N \times A \times T$ for $N$ operations, $A$ addresses, and $T$
threads to $2 \times N \times T$.  Not only does this save space, but
it makes the backwards analysis faster as the amount of information
being propagated is smaller.  In other words, the efficiency of our
checker depends on the number of different address locations used in
the trace.  This is not the case for Manovit's TSO-only checker.


\section{POWER model (POW)}
\label{Section:POWERModel}

The key relaxation introduced by the POW model is to allow writes to
be observed by some threads before they can be observed by others
(known as ``non-multi-copy-atomicity'' \cite{POWER}).  This supports
(though is not necessary for) cache hierarchies involving more than
one shared cache.  Before presenting the model, we look at a few
examples demonstrating this relaxed behaviour.

\paragraph{Example (WRC+deps)}  The following trace is allowed by POW
but forbidden by WMO.

\begin{verbatim}
0: M[0] := 1
1: M[0] == 1    @ 100:110
1: M[1] := 1    @ 115
2: M[1] == 1    @ 200:210
2: M[0] == 0    @ 215
\end{verbatim}

\noindent Notice the dependencies prevent any local reordering of
operations, which is why the trace is forbidden by WMO.  This is an
example of non-multi-copy-atomic behaviour: the write by thread 0
propagates to thread 1 before it propagates to thread 2.  At the
hardware level, this could be explained by the presence of a cache
shared by threads 0 and 1 but not 2.

\paragraph{Example (WRC+sync+dep)} A single \verb'sync' operation,
inserted as follows, is enough to forbid the relaxed behaviour.

\begin{verbatim}
0: M[0] := 1
1: M[0] == 1
1: sync
1: M[1] := 1
2: M[1] == 1    @ 200:210
2: M[0] == 0    @ 215
\end{verbatim}

\noindent This demonstrates the so-called ``cumulative'' property of
\verb!sync! \cite{POWER}: not only does \verb!sync! ensure that all
preceding writes by the issuing thread have propagated to all other
threads, but it also ensures that any writes that the issuing thread
has observed before the \verb!sync! have also propagated.  In this
example, any thread that sees the write by thread 1 must also see the
write by thread 0 because thread 1's write is preceded by a
\verb'sync' which is in turn preceded by an observation of thread 0's
write.

\paragraph{Example (SB+syncs and MP+sync+dep revisited)} Under POW,
these examples are still forbidden: the \verb!sync! instructions are
still are enough to forbid the relaxed behaviour.

\paragraph{Example (WWC+deps)} The following trace is allowed by POW
but forbidden by WMO.

\begin{verbatim}
0: M[0] := 1
1: M[0] == 1 @ 100:110
1: M[1] := 1 @ 115:
2: M[1] == 1 @ 200:210
2: M[0] := 2 @ 215:
final M[0] == 1
\end{verbatim}

\noindent At the hardware level, this example can again be be
explained by the presence of a cache, say $C$, shared by threads 0 and
1 but not 2: (1) the write by thread 0 can reach $C$ and be
observed by thread 1; (2) $C$ can evict the write by thread 1 before
the write by thread 0; (3) thread 2 can observe the write of thread 1
in the last-level cache; (4) the write by thread 2 can reach the
last-level cache; and (5) $C$ can evict the write by thread 0 and
overwrite thread 2's write in the last-level cache.

\subsection*{Axiomatic definition}

We first present a semantics that ignores atomic read-modify-write
operations, and then extend the semantics to support them.  The
semantics is defined by the combination of two kinds of order:

\begin{itemize}

\item For each address $a$, a value order $<_a$ over values written
to address a.

\item An operation order $\prec$ over operations in the trace.

\end{itemize}

\noindent Let $val(op)$, where $op$ is a load or store, denote the value
read or written by $op$.

\noindent Let $rf(op)$, where $op$ is a load, denote the store
operation in the trace with the same (address,value) pair as $op$.

\noindent If $op$ is the first operation to address $a$ on some thread and
$val(op) \neq 0$ then add constraint $0 <_a val(op)$.

\noindent If $op_1$ and $op_2$ are thread-ordered operations that
access address $a$ and $val(op_1) \neq val(op_2)$ then add constraint
$val(op_1) <_a val(op_2)$.

\noindent Let $op_1$ and $op_2$ be thread-ordered operations.
Add constraint $op_1 \prec op_2$ if:

\begin{itemize}

\item $op_1$ is a load and $op_2$ accesses the same address; or

\item $op_1$ and $op_2$ are stores to the same address; or

\item $op_1$ is a \verb!sync! or $op_2$ is a \verb!sync!; or

\item $op_1$ is a load with end-time $t_1$ and $op_2$ has begin-time
$t_2$ and $t_1 < t_2$.

\end{itemize}

\noindent (Note: these are the same as the local axioms for the WMO
model.)

\noindent For each load $op$, add constraint $rf(op) \prec op$.

\noindent Let $op_1$ and $op_2$ be any two distinct \verb!sync!
operations from the trace.  Add constraint
$x \prec y \vee y \prec x$, i.e. a total-order over syncs.

\noindent Let $op_1$ and $op_2$ be two distinct \verb!sync! operations
from the trace. For each address $a$, add constraint
$op_1 \prec op_2 \Rightarrow v <_a w$ if $v \neq w$ where:
\begin{itemize}
\item
  $v$ is the latest value seen at address $a$ before $op_1$
      in thread-order; and
\item
  $w$ is the earliest value seen at
      address $a$ after $op_2$ in thread-order.
\end{itemize}

\noindent Let $op_1$ be a \verb!sync! and $op_2$ be a load with
response-time $t$. Let $op_3$ be the first operation with a
request-time larger then $t$ that follows $op_2$ in thread order.
For each address $a$, add constraint $op_1 \prec op_2 \Rightarrow v <_a
w$ if $v \neq w$ where:
\begin{itemize}
\item
  $v$ is the latest value seen at address $a$ before $op_1$
      in thread-order; and
\item
  $w$ is the earliest value seen at
      address $a$ at or after after $op_3$ in thread-order.
\end{itemize}

\noindent If there is a solution to all the above constraints, then
the trace conforms to the POW model, otherwise it doesn't.

\subsection*{Atomic operations}

Atomic read-modify-write operations are treated simply as adjacent
thread-ordered read and write operations in the above semantics,
provided the following condition is met: for each address $a$ there
must exist a topological sort of $<_a$ in which $v$ and $w$ are
adjacent for each read-modify-write operation
$\texttt{<M[}a\texttt{]==}~v\texttt{; M[}a\texttt{]:=}~w\texttt{>}$.

\subsection{Checking algorithm}

Our checking algorithm for POW is essentially a solver for the above
constraint set.  When the \verb!-g! option is specified (indicating a
global clock domain), we infer an additional ordering between any two
\texttt{sync} operations running on different threads if one has a
response time that precedes the request time of the other.  This can
significantly reduce the amount of non-determinism in the solver.

\section{Performance}
\label{Section:Performance}

\begin{figure}
\begin{center}
\includegraphics{performance/tso.pdf}
\end{center}
\caption{Performance of the TSO checker.}
\label{Graph:TSO}
\end{figure}

\begin{figure}
\begin{center}
\includegraphics{performance/wmo.pdf}
\end{center}
\caption{Performance of the WMO checker.}
\label{Graph:WMO}
\end{figure}

\begin{figure}
\begin{center}
\includegraphics{performance/pow.pdf}
\end{center}
\caption{Performance of the POW checker with the \texttt{-g} flag specified.}
\label{Graph:POW}
\end{figure}



For performance evaluation, we have generated a range of
traces\footnote{Using a model cache implementation 
with load and store buffering, out-of-order eviction,
out-of-order responses, prefetching and invalidation-based coherence.}
with various numbers of memory operations ($n \in \{8K, 16K, 24K,
32K\})$, threads ($t \in \{4, 16, 32\}$), and addresses ($a \in \{4,
16, 32\}$).  For each combination of parameters, we generate 16
traces, giving 576 traces for each supported model.

Figures \ref{Graph:TSO}, \ref{Graph:WMO}, and \ref{Graph:POW} show how
the performances of the TSO, WMO and POW checkers vary with the number
of operations and threads present, averaged over the number of
addresses present.  In the case of the POW checker, we use the
\verb!-g! flag, specifying a global clock domain and 
enabling the use of begin and end times to infer a partial global
ordering of \verb!sync! operations.  Without the \verb!-g! flag, the
POW checker has a limited completion rate: for 4, 16 and 32 threads
respectively, it has a completion rate of 100\%, 96\%, and 54\%.

\section{Correctness}
\label{Section:Correctness}

Axe has been tested for equivalence against an operational semantics
for each model and also an axiomatic semantics for each model
(\S\ref{Section:SPARCModels} and \S\ref{Section:POWERModel}).  The
test traces include: (1) 199 litmus tests from the PPCMEM distribution
\cite{PPCMEM}; and (2) 200K randomly-generated traces ranging from
around 10 to 50 operations in size.  For the litmus tests, Axe's POW
checker gives the same outcome as PPCMEM.  Axe also gives the expected
outcomes for all the traces used in our performance evaluation
(\S\ref{Section:Performance}).  

The 199 litmus test traces and the 200,000 randomly generated traces are
both distributed with the Axe tool (in the \verb!tests!
sub-directory), along with the expected outcomes of each trace on each
model.  In addition, the outcomes of the litmus test traces are listed
in Appendix B.

\section*{Acknowledgements}

Thanks to members of the Semantics group at the University of
Cambridge Computer Laboratory for numerous clarifications about
relaxed memory models.  This work was supported by DARPA/AFRL
contracts FA8750-10-C-0237 (CTSRD) and FA8750-11-C-0249 (MRC2), and
EPSRC grant EP/K008528/1 (REMS).  The views, opinions, and/or findings
contained in this manual are those of the authors and should not be
interpreted as representing the official views or policies, either
expressed or implied, of the Department of Defense or the U.S.
Government.

\begin{thebibliography}{99}
\addcontentsline{toc}{section}{References}
\setlength{\itemsep}{1pt}

\bibitem{BlueCheck} M. Naylor and S. W. Moore, \emph{A generic
synthesisable test bench}, in MEMOCODE 2015, pp. 128--137.

\bibitem{Litmus} J. Alglave, L. Maranget, S. Sarkar, and P.
Sewell, \emph{Litmus: Running Tests Against Hardware}, in TACAS 2011,
pp. 41--44.

\bibitem{SC} L. Lamport.  \emph{How to Make a Multiprocessor Computer
That Correctly Executes Multiprocess Programs}, IEEE Transactions on
Computers, volume 28, number 9, pp. 690--691, 1979.

\bibitem{SPARC} D. L. Weaver and T. Germond. \emph{ The SPARC
Architecture Manual Version 9}, 2003.

\bibitem{POWER} S. Sarkar, P. Sewell, J.  Alglave, L. Maranget, and D.
Williams. \emph{Understanding POWER Multiprocessors}, SIGPLAN Notices,
vol. 46, num. 6, pp. 175--186, June 2011.

\bibitem{CHERI} \emph{Homepage of the CHERI processor (Capability
Hardware Enhanced RISC Instructions)}, \url{http://chericpu.org}.

\bibitem{Gibbons} P. B.  Gibbons and E. Korach.  \emph{On testing
cache-coherent shared memories}, in SPAA 1994, pp.  177–-188.

\bibitem{Manovit} C. Manovit. \emph{Testing memory consistency of
shared-memory multiprocessors}, PhD thesis, Stanford University, 2006.

\bibitem{TSOToolISCA} S. Hangal, D. Vahia, C. Manovit, and JY. J. Lu,
\emph{TSOtool: A Program for Verifying Memory Systems Using the Memory
Consistency Model}, in ISCA 2004, pp. 114.

\bibitem{TSOToolSPAA} C. Manovit and S. Hangal, \emph{Efficient
algorithms for verifying memory consistency}, in SPAA 2005, pp.
245--252.

\bibitem{TSOToolHPCA} C. Manovit and S. Hangal, \emph{Completely
verifying memory consistency of test program executions}, in HPCA
2006, pp. 166--175.

\bibitem{TSOTool} \emph{Homepage of TSOTool}, a program for verifying
memory systems using the memory consistency model,
\url{http://xenon.stanford.edu/~hangal/tsotool.html}.

\bibitem{IntelChecker} A. Roy, S. Zeisset, C. J. Fleckenstein, J. C.
Huang, \emph{Fast and Generalized Polynomial Time Memory Consistency
Verification}, CAV 2006, pp. 503.%--516.

\bibitem{ParkDill} S. Park and D. L. Dill.  \emph{ An executable
specification, analyzer and verifier for RMO (relaxed memory order)},
in SPAA 1995.

\bibitem{PPCMEM} \emph{Homepage of PPCMEM/ARMMEM},
a tool for exploring the POWER and ARM memory models,
\url{https://www.cl.cam.ac.uk/~pes20/ppcmem/}.

\bibitem{RocketChip} K. Asanovic et al., \emph{The Rocket Chip
Generator}, Technical Report UCB/EECS-2016-17, University of
California, Berkeley, 2016.
%,\url{https://github.com/ucb-bar/rocket-chip}.

%\bibitem{Herd} J. Alglave, L. Maranget, M. Tautschnig. \emph{Herding
%Cats: Modelling, Simulation, Testing, and Data Mining for Weak
%Memory}, ACM TOPLAS, vol. 36, num. 2, pp. 7:1--7:74, July 2014.


%\bibitem{MemModels} S. Adve and K. Gharachorloo. \emph{Shared Memory
%Consistency Models: A Tutorial}, Computer Journal, volume 29, number
%12, pp. 66--76, 1996.


\end{thebibliography}

\section*{Appendix A: Example applications}
\addcontentsline{toc}{section}{Appendix A: Example applications}

\subsection*{Berkeley's Rocket Chip}
\addcontentsline{toc}{subsection}{Berkeley's Rocket Chip}

Rocket Chip is an open-source system-on-chip generator developed at UC
Berkeley including support for multiple processor cores and a
cache-coherent shared memory subsystem.  Available processor cores
include the in-order Rocket, the out-of-order BOOM, the Z-scale
microcontroller, and (pending release) the Hwacha vector-thread
accelerator.  Having been taped out 11 times between 2011 and 2015,
Rocket Chip is fairly mature but faces constant change through
extensions, redesigns, and refactorings.  Rocket Chip is written using
the Chisel HDL, also from UC Berkeley.

The Rocket Chip developers have recognised the importance of
making HDL-level test benches for the memory subsystem: \emph{``In
order to test behaviors in our memory hierarchy which are not easy or
efficient to test in software, we have designed a set of test circuits
called GroundTest''} \cite{RocketChip}.  GroundTest plugs into the
socket given to CPU tiles and generates various kinds of memory
traffic directly to the memory subsystem, either via the L1 caches, or
directly to the L2, or via DMA.

Rocket Chip is highly parameterised, and this includes the choice of
coherence protocol which by default (at the time of writing) is MESI.
Since MESI guarantees either a single writer or multiple readers to a
cache line at any time, it gives the illusion of a single shared
memory -- despite the reality of multiple local caches -- and is thus
expected to conform to one of the SPARC consistency models.

\paragraph{Extending GroundTest}

We developed a \emph{trace generator} that plugs into the
GroundTest framework.  Given a random seed, it generates random
memory requests from each tile, and emits a trace of events.  To
illustrate, here is an example of generated trace.

\begin{verbatim}
  1: load-req     0x0000000008 #0 @64
  1: store-req  5 0x0000100008 #1 @65
  1: store-req  7 0x0000000010 #2 @66
  0: store-req  2 0x0000000008 #0 @303
  0: load-req     0x0000000008 #1 @304
  0: store-req  6 0x0000100008 #2 @305
  1: resp       0              #0 @96
  0: resp       0              #0 @350
  0: resp       2              #1 @351
  0: load-req     0x0000000010 #3 @353
  1: resp       0              #1 @149
  1: load-req     0x0000000108 #3 @152
  1: resp       0              #3 @184
  0: resp       5              #2 @422
  0: resp       0              #3 @424
  1: resp       0              #2 @226
\end{verbatim}

\noindent Main syntactical points:

\begin{itemize}

\item the first number on each line of the trace is the
\textbf{thread-id};

\item $\verb!#!n$ denotes a \textbf{request-id} $n$;

\item $\verb!@!t$ denotes a \textbf{time} $t$ in clock cycles;

\item hex numbers denote \textbf{addresses};

\item remaining decimal numbers denote \textbf{values} being loaded or stored;

\item this trace contains only \textbf{loads}, \textbf{stores} and
\textbf{responses}, but the generator also supports \textbf{LR/SC}
pairs, \textbf{atomics}, and \textbf{fences}.

\end{itemize}

Notice that the timestamps are not monotonically increasing.  In
simulation, the Rocket Chip tiles are brought out of reset
sequentially -- each tile starts roughly 250 cycles apart.  Hence the
timestamps are local to each thread, which is not a problem for Axe
but may be confusing when reading the trace.

The number of tiles, requests, and addresses used when generating a
trace can all be controlled using \emph{compile-time} parameters.
Ideally though, the number of requests would be taken as a
\emph{simulation-time} parameter, allowing an iterative-deepening
strategy in which the trace generator is repeatedly invoked to emit
gradually longer traces over time, in the hope of finding simple
failures first.  Unfortunately, we are not aware of a convenient way
to do this in the Chisel HDL; the equivalent of Verilog's
\verb!$value$plusargs! or \verb!$fscanf! would have been very helpful
here.

Another compile-time parameter that would ideally be a simulation-time
parameter is the set of memory addresses used by each tile: varying
the address set by simply rerunning the trace generator would make it
easier to explore a wide range of address combinations.  The problem
is that each tile needs to know at least a subset of the addresses
being used by other tiles (otherwise the chances of generating shared
variables would be very low) but we are not aware of a convenient way
for tiles to share simulation-time data in GroundTest.  Again, a
Verilog-style \verb!$fscanf! function would suffice to overcome this
problem: the common address set could be read from a file by each
tile.

Finding a clean way to terminate the Rocket Chip simulator on
completion of the trace generator also proved to be a challenge.  The
standard exit strategy is to raise a flag in one of Rocket Chip's CSR
registers but that causes the entire simulation to stop even though
some tiles may not yet have finished -- and there is no convenient way
for tiles to inform each other that they have finished.  We settled
for the following solution: when a tile finishes it emits a special
``finished'' message in the trace.  A wrapper script runs the trace
generator, waits until all tiles have finished, and then sends a
\verb!SIGTERM! to the simulator (which gracefully handles this
signal).

\paragraph{Converting traces to Axe format} We made a simple script to
convert traces emitted by the trace generator into Axe format.  For
example, given the above sample trace, this conversion script yields:

\begin{verbatim}
  # &M[2] == 0x0000000010
  # &M[0] == 0x0000000008
  # &M[3] == 0x0000000108
  # &M[1] == 0x0000100008
  1: M[0] == 0 @ 64:96
  1: M[1] := 5 @ 65:
  1: M[2] := 7 @ 66:
  0: M[0] := 2 @ 303:
  0: M[0] == 2 @ 304:351
  0: M[1] := 6 @ 305:
  0: M[2] == 0 @ 353:424
  1: M[3] == 0 @ 152:184
\end{verbatim}

\noindent Notice that lines beginning with \verb!#! are treated as
comments by Axe: we use these comments to record the mapping between
physical addresses and addresses used by Axe.

\paragraph{Testing against the SC model} We made a script
that repeatedly: (1) generates a trace with a random seed; (2)
converts the trace to Axe format; and (3) checks the trace against the
chosen consistency model.  Running this script, we found a 260-element
trace that fails to satisfy sequential consistency.  Passing this
trace through \verb!axe-shrink.py!  (see \S\ref{Section:CmdLine}), we
get:

\begin{verbatim}
  Pass 0
  Omitted 245 of 260         
  Pass 1
  Omitted 255 of 260         
  Pass 2
  Omitted 255 of 260         
  1: M[1] := 185 @ 1921:
  1: M[0] := 193 @ 1966:
  0: M[0] == 193 @ 2207:2245
  0: M[1] := 204 @ 2208:
  0: M[1] == 185 @ 2209:2269
\end{verbatim}

\noindent This trace (similar to the MP example) can be explained
either by thread 1's stores being performed out-of-order (PSO) or
thread 0's loads being performed out-of-order (WMO).

\paragraph{Testing against the PSO model}  We also found a
261-element trace that violates PSO. 

\begin{verbatim}
  Pass 0
  Omitted 252 of 261         
  Pass 1
  Omitted 257 of 261         
  Pass 2
  Omitted 257 of 261         
  0: M[2] == 137 @ 1825:1948
  0: M[0] := 154 @ 1886:
  1: M[0] == 154 @ 1689:1725
  1: M[2] := 137 @ 1690:
\end{verbatim}

\noindent This trace (similar to the LB example) can be explained by
the load and store on thread 0 (or thread 1) being reordered (WMO).

\paragraph{Coherence bug} We observed a large number
of traces that satisfy the WMO model, but eventually hit this
counter-example:

\begin{verbatim}
  Pass 0
  Omitted 241 of 260         
  Pass 1
  Omitted 256 of 260         
  Pass 2
  Omitted 256 of 260         
  0: M[2] := 46 @ 497:
  1: M[2] == 46 @ 280:513
  1: M[2] := 61 @ 729:
  1: M[2] == 46 @ 854:979
\end{verbatim}

\noindent Note that the write of \verb!M[2] := 46! by core 0 is the
only write of 46 in the entire trace (the trace generator ensures that
all write values are unique). Also, the initial value of each location
is 0.  Therefore, the write \verb!M[2] := 61!  by core 1 has seemingly
been dropped.  
This is a coherence violation and undesirable: if the
write of 46 to \verb!M[2]!  is interpreted as ``core 1, a message is
available'' then core 1 might end up receiving two messages as it
effectively sees the write twice.
(It sees 46 once on line 2, then it
clears that value with a store on line 3, and finally it sees 46 again
on line 4).

We reported this issue to the Rocket Chip developers who identified a
race condition in the coherence protocol and fixed it within a few
days.

\paragraph{Livelock bug} For the above testing we only enabled loads
and stores in the trace generator.  When we enabled generation of
LR/SC pairs, we found a lock-up issue in which a store-conditional
would never return under some circumstances.  We reported this to the
Rocket Chip developers who diagnosed the problem as a livelock issue
in the coherence protocol.

\paragraph{Store-conditional bug}  With the livelock issue fixed,
we found the following counter-example to WMO:

\begin{verbatim}
  Pass 0
  Omitted 217 of 228         
  Pass 1
  Omitted 224 of 228         
  Pass 2
  Omitted 224 of 228         
  1: M[3] := 31 @ 340:
  0: { M[3] == 31; M[3] := 178} @ 745:812
  0: { M[3] == 178; M[3] := 198} @ 926:955
  1: { M[3] == 178; M[3] := 59} @ 759:761
\end{verbatim}

\noindent Notice that the read-modify-write by thread 1 atomically
changes \verb!M[0]! from 178 to 59.  Furthermore, the second
read-modify-write on thread 0 atomically changes \verb!M[0]! from 178
to 198.  Of course, if these operations really were atomic, this
behaviour would be impossible.  After investigating the raw trace
emitted by the trace generator, we noticed this issue arises when a
store-conditional is issued before a load-reserve response is
received.

We reported this issue to the Rocket Chip developers who
identified it as a bug in which a cache line is not marked as dirty
when it should.

\paragraph{Testing against the WMO model} At the time of writing --
with loads, stores, LR/SC pairs, atomics, and fences all being
generated -- Rocket Chip satisfies the WMO model on thousands of large
traces (32k operations per trace, 16 addresses, 8 threads).

\paragraph{Liveness} A key limitation of the above specification-based
testing approach is that it does not check for liveness, e.g. that a
store-conditional operation succeeds when it should.  In response, we
added a mode to the trace generator in which it will only generate
LR/SC pairs that are expected to succeed.  This is possible in Rocket
Chip because of the way it implements LR/SC: the L1 cache will hold on
to a cache line for a maximum of $n$ cycles after an LR response.  So
provided the LR and SC are within $n$ cycles of each other, the SC
should succeed.  In this mode, we observed an LR/SC success rate of
94\%.  The 6\% of failures remain unexplained and we plan to explore
this in future work.

\section*{Appendix B: Litmus test results}
\addcontentsline{toc}{section}{Appendix B: Litmus test results}

The following table gives the output of Axe on a large number of
litmus tests from the PPCMEM distribution.  These tests can also be
found in the Axe distribution in the \verb#tests/litmus# subdirectory.
We use a tick mark (\cmark) to denote that the behaviour described by
the test is allowed, and an empty cell to denote that it is forbidden.
In each case, our POW model gives the same outcome as PPCMEM.  

\begin{longtable}{lccccc}
\textbf{Test name} & \textbf{SC} & \textbf{TSO} &
\textbf{PSO} & \textbf{WMO} & \textbf{POW} \\
\\
\endhead
\texttt{2+2W+sync+po } &  &  & \cmark & \cmark & \cmark \\
\texttt{3.2W } &  &  & \cmark & \cmark & \cmark \\
\texttt{3.2W+sync+po+po } &  &  & \cmark & \cmark & \cmark \\
\texttt{3.2W+syncs } &  &  &  &  &  \\
\texttt{3.2W+sync+sync+po } &  &  & \cmark & \cmark & \cmark \\
\texttt{3.LB+addr+addr+po } &  &  &  & \cmark & \cmark \\
\texttt{3.LB+addr+po+po } &  &  &  & \cmark & \cmark \\
\texttt{3.LB+addrs } &  &  &  &  &  \\
\texttt{3.LB+addr+sync+po } &  &  &  & \cmark & \cmark \\
\texttt{3.LB } &  &  &  & \cmark & \cmark \\
\texttt{3.LB+sync+addr+addr } &  &  &  &  &  \\
\texttt{3.LB+sync+addr+po } &  &  &  & \cmark & \cmark \\
\texttt{3.LB+sync+po+po } &  &  &  & \cmark & \cmark \\
\texttt{3.LB+syncs } &  &  &  &  &  \\
\texttt{3.LB+sync+sync+addr } &  &  &  &  &  \\
\texttt{3.LB+sync+sync+po } &  &  &  & \cmark & \cmark \\
\texttt{3.SB } &  & \cmark & \cmark & \cmark & \cmark \\
\texttt{3.SB+sync+po+po } &  & \cmark & \cmark & \cmark & \cmark \\
\texttt{3.SB+syncs } &  &  &  &  &  \\
\texttt{3.SB+sync+sync+po } &  & \cmark & \cmark & \cmark & \cmark \\
\texttt{IRIW+addr+po } &  &  &  & \cmark & \cmark \\
\texttt{IRIW+addrs } &  &  &  &  & \cmark \\
\texttt{IRIW } &  &  &  & \cmark & \cmark \\
\texttt{IRIW+sync+addr } &  &  &  &  & \cmark \\
\texttt{IRIW+sync+po } &  &  &  & \cmark & \cmark \\
\texttt{IRIW+syncs } &  &  &  &  &  \\
\texttt{IRRWIW+addr+po } &  &  &  & \cmark & \cmark \\
\texttt{IRRWIW+addrs } &  &  &  &  & \cmark \\
\texttt{IRRWIW+addr+sync } &  &  &  &  & \cmark \\
\texttt{IRRWIW } &  &  &  & \cmark & \cmark \\
\texttt{IRRWIW+po+addr } &  &  &  & \cmark & \cmark \\
\texttt{IRRWIW+po+sync } &  &  &  & \cmark & \cmark \\
\texttt{IRRWIW+sync+addr } &  &  &  &  & \cmark \\
\texttt{IRRWIW+sync+po } &  &  &  & \cmark & \cmark \\
\texttt{IRRWIW+syncs } &  &  &  &  &  \\
\texttt{IRWIW+addr+po } &  &  &  & \cmark & \cmark \\
\texttt{IRWIW+addrs } &  &  &  &  & \cmark \\
\texttt{IRWIW } &  &  &  & \cmark & \cmark \\
\texttt{IRWIW+sync+addr } &  &  &  &  & \cmark \\
\texttt{IRWIW+sync+po } &  &  &  & \cmark & \cmark \\
\texttt{IRWIW+syncs } &  &  &  &  &  \\
\texttt{ISA2+sync+addr+addr } &  &  &  &  &  \\
\texttt{ISA2+sync+addr+po } &  &  &  & \cmark & \cmark \\
\texttt{ISA2+sync+addr+sync } &  &  &  &  &  \\
\texttt{ISA2+sync+po+addr } &  &  &  & \cmark & \cmark \\
\texttt{ISA2+sync+po+po } &  &  &  & \cmark & \cmark \\
\texttt{ISA2+sync+po+sync } &  &  &  & \cmark & \cmark \\
\texttt{ISA2+syncs } &  &  &  &  &  \\
\texttt{ISA2+sync+sync+addr } &  &  &  &  &  \\
\texttt{ISA2+sync+sync+po } &  &  &  & \cmark & \cmark \\
\texttt{LB+addr+po } &  &  &  & \cmark & \cmark \\
\texttt{LB+addrs } &  &  &  &  &  \\
\texttt{LB } &  &  &  & \cmark & \cmark \\
\texttt{LB+sync+addr } &  &  &  &  &  \\
\texttt{LB+sync+po } &  &  &  & \cmark & \cmark \\
\texttt{LB+syncs } &  &  &  &  &  \\
\texttt{MP } &  &  & \cmark & \cmark & \cmark \\
\texttt{MP+po+addr } &  &  & \cmark & \cmark & \cmark \\
\texttt{MP+po+sync } &  &  & \cmark & \cmark & \cmark \\
\texttt{MP+sync+addr } &  &  &  &  &  \\
\texttt{MP+sync+po } &  &  &  & \cmark & \cmark \\
\texttt{MP+syncs } &  &  &  &  &  \\
\texttt{R } &  & \cmark & \cmark & \cmark & \cmark \\
\texttt{R+po+sync } &  &  & \cmark & \cmark & \cmark \\
\texttt{R+sync+po } &  & \cmark & \cmark & \cmark & \cmark \\
\texttt{R+syncs } &  &  &  &  &  \\
\texttt{RWC+addr+po } &  & \cmark & \cmark & \cmark & \cmark \\
\texttt{RWC+addr+sync } &  &  &  &  & \cmark \\
\texttt{RWC } &  & \cmark & \cmark & \cmark & \cmark \\
\texttt{RWC+po+sync } &  &  &  & \cmark & \cmark \\
\texttt{RWC+sync+po } &  & \cmark & \cmark & \cmark & \cmark \\
\texttt{RWC+syncs } &  &  &  &  &  \\
\texttt{S } &  &  & \cmark & \cmark & \cmark \\
\texttt{SB } &  & \cmark & \cmark & \cmark & \cmark \\
\texttt{SB+sync+po } &  & \cmark & \cmark & \cmark & \cmark \\
\texttt{SB+syncs } &  &  &  &  &  \\
\texttt{S+po+addr } &  &  & \cmark & \cmark & \cmark \\
\texttt{S+po+sync } &  &  & \cmark & \cmark & \cmark \\
\texttt{S+sync+addr } &  &  &  &  &  \\
\texttt{S+sync+po } &  &  &  & \cmark & \cmark \\
\texttt{S+syncs } &  &  &  &  &  \\
\texttt{WRC+addr+po } &  &  &  & \cmark & \cmark \\
\texttt{WRC+addrs } &  &  &  &  & \cmark \\
\texttt{WRC+addr+sync } &  &  &  &  & \cmark \\
\texttt{WRC } &  &  &  & \cmark & \cmark \\
\texttt{WRC+po+addr } &  &  &  & \cmark & \cmark \\
\texttt{WRC+po+sync } &  &  &  & \cmark & \cmark \\
\texttt{WRC+sync+addr } &  &  &  &  &  \\
\texttt{WRC+sync+po } &  &  &  & \cmark & \cmark \\
\texttt{WRC+syncs } &  &  &  &  &  \\
\texttt{WRR+2W+addr+po } &  &  & \cmark & \cmark & \cmark \\
\texttt{WRR+2W+addr+sync } &  &  &  &  & \cmark \\
\texttt{WRR+2W } &  &  & \cmark & \cmark & \cmark \\
\texttt{WRR+2W+po+sync } &  &  &  & \cmark & \cmark \\
\texttt{WRR+2W+sync+po } &  &  & \cmark & \cmark & \cmark \\
\texttt{WRR+2W+syncs } &  &  &  &  &  \\
\texttt{WRW+2W+addr+po } &  &  & \cmark & \cmark & \cmark \\
\texttt{WRW+2W+addr+sync } &  &  &  &  & \cmark \\
\texttt{WRW+2W } &  &  & \cmark & \cmark & \cmark \\
\texttt{WRW+2W+po+sync } &  &  &  & \cmark & \cmark \\
\texttt{WRW+2W+sync+po } &  &  & \cmark & \cmark & \cmark \\
\texttt{WRW+2W+syncs } &  &  &  &  &  \\
\texttt{W+RWC } &  & \cmark & \cmark & \cmark & \cmark \\
\texttt{W+RWC+po+addr+po } &  & \cmark & \cmark & \cmark & \cmark \\
\texttt{W+RWC+po+addr+sync } &  &  & \cmark & \cmark & \cmark \\
\texttt{W+RWC+po+po+sync } &  &  & \cmark & \cmark & \cmark \\
\texttt{W+RWC+po+sync+po } &  & \cmark & \cmark & \cmark & \cmark \\
\texttt{W+RWC+po+sync+sync } &  &  & \cmark & \cmark & \cmark \\
\texttt{W+RWC+sync+addr+po } &  & \cmark & \cmark & \cmark & \cmark \\
\texttt{W+RWC+sync+addr+sync } &  &  &  &  &  \\
\texttt{W+RWC+sync+po+po } &  & \cmark & \cmark & \cmark & \cmark \\
\texttt{W+RWC+sync+po+sync } &  &  &  & \cmark & \cmark \\
\texttt{W+RWC+syncs } &  &  &  &  &  \\
\texttt{W+RWC+sync+sync+po } &  & \cmark & \cmark & \cmark & \cmark \\
\texttt{WRW+WR+addr+po } &  & \cmark & \cmark & \cmark & \cmark \\
\texttt{WRW+WR+addr+sync } &  &  &  &  & \cmark \\
\texttt{WRW+WR } &  & \cmark & \cmark & \cmark & \cmark \\
\texttt{WRW+WR+po+sync } &  &  &  & \cmark & \cmark \\
\texttt{WRW+WR+sync+po } &  & \cmark & \cmark & \cmark & \cmark \\
\texttt{WRW+WR+syncs } &  &  &  &  &  \\
\texttt{WWC+addr+po } &  &  &  & \cmark & \cmark \\
\texttt{WWC+addrs } &  &  &  &  & \cmark \\
\texttt{WWC+addr+sync } &  &  &  &  & \cmark \\
\texttt{WWC } &  &  &  & \cmark & \cmark \\
\texttt{WWC+po+addr } &  &  &  & \cmark & \cmark \\
\texttt{WWC+po+sync } &  &  &  & \cmark & \cmark \\
\texttt{WWC+sync+addr } &  &  &  &  &  \\
\texttt{WWC+sync+po } &  &  &  & \cmark & \cmark \\
\texttt{WWC+syncs } &  &  &  &  &  \\
\texttt{Z6.0 } &  & \cmark & \cmark & \cmark & \cmark \\
\texttt{Z6.0+po+addr+po } &  & \cmark & \cmark & \cmark & \cmark \\
\texttt{Z6.0+po+addr+sync } &  &  & \cmark & \cmark & \cmark \\
\texttt{Z6.0+po+po+sync } &  &  & \cmark & \cmark & \cmark \\
\texttt{Z6.0+po+sync+po } &  & \cmark & \cmark & \cmark & \cmark \\
\texttt{Z6.0+po+sync+sync } &  &  & \cmark & \cmark & \cmark \\
\texttt{Z6.0+sync+addr+po } &  & \cmark & \cmark & \cmark & \cmark \\
\texttt{Z6.0+sync+addr+sync } &  &  &  &  &  \\
\texttt{Z6.0+sync+po+po } &  & \cmark & \cmark & \cmark & \cmark \\
\texttt{Z6.0+sync+po+sync } &  &  &  & \cmark & \cmark \\
\texttt{Z6.0+syncs } &  &  &  &  &  \\
\texttt{Z6.0+sync+sync+po } &  & \cmark & \cmark & \cmark & \cmark \\
\texttt{Z6.1 } &  &  & \cmark & \cmark & \cmark \\
\texttt{Z6.1+po+po+addr } &  &  & \cmark & \cmark & \cmark \\
\texttt{Z6.1+po+po+sync } &  &  & \cmark & \cmark & \cmark \\
\texttt{Z6.1+po+sync+addr } &  &  & \cmark & \cmark & \cmark \\
\texttt{Z6.1+po+sync+po } &  &  & \cmark & \cmark & \cmark \\
\texttt{Z6.1+po+sync+sync } &  &  & \cmark & \cmark & \cmark \\
\texttt{Z6.1+sync+po+addr } &  &  & \cmark & \cmark & \cmark \\
\texttt{Z6.1+sync+po+po } &  &  & \cmark & \cmark & \cmark \\
\texttt{Z6.1+sync+po+sync } &  &  & \cmark & \cmark & \cmark \\
\texttt{Z6.1+syncs } &  &  &  &  &  \\
\texttt{Z6.1+sync+sync+addr } &  &  &  &  &  \\
\texttt{Z6.1+sync+sync+po } &  &  &  & \cmark & \cmark \\
\texttt{Z6.2 } &  &  & \cmark & \cmark & \cmark \\
\texttt{Z6.2+po+addr+addr } &  &  & \cmark & \cmark & \cmark \\
\texttt{Z6.2+po+addr+po } &  &  & \cmark & \cmark & \cmark \\
\texttt{Z6.2+po+addr+sync } &  &  & \cmark & \cmark & \cmark \\
\texttt{Z6.2+po+po+addr } &  &  & \cmark & \cmark & \cmark \\
\texttt{Z6.2+po+po+sync } &  &  & \cmark & \cmark & \cmark \\
\texttt{Z6.2+po+sync+addr } &  &  & \cmark & \cmark & \cmark \\
\texttt{Z6.2+po+sync+po } &  &  & \cmark & \cmark & \cmark \\
\texttt{Z6.2+po+sync+sync } &  &  & \cmark & \cmark & \cmark \\
\texttt{Z6.2+sync+addr+addr } &  &  &  &  &  \\
\texttt{Z6.2+sync+addr+po } &  &  &  & \cmark & \cmark \\
\texttt{Z6.2+sync+addr+sync } &  &  &  &  &  \\
\texttt{Z6.2+sync+po+addr } &  &  &  & \cmark & \cmark \\
\texttt{Z6.2+sync+po+po } &  &  &  & \cmark & \cmark \\
\texttt{Z6.2+sync+po+sync } &  &  &  & \cmark & \cmark \\
\texttt{Z6.2+syncs } &  &  &  &  &  \\
\texttt{Z6.2+sync+sync+addr } &  &  &  &  &  \\
\texttt{Z6.2+sync+sync+po } &  &  &  & \cmark & \cmark \\
\texttt{Z6.3 } &  &  & \cmark & \cmark & \cmark \\
\texttt{Z6.3+po+po+addr } &  &  & \cmark & \cmark & \cmark \\
\texttt{Z6.3+po+po+sync } &  &  & \cmark & \cmark & \cmark \\
\texttt{Z6.3+po+sync+addr } &  &  & \cmark & \cmark & \cmark \\
\texttt{Z6.3+po+sync+po } &  &  & \cmark & \cmark & \cmark \\
\texttt{Z6.3+po+sync+sync } &  &  & \cmark & \cmark & \cmark \\
\texttt{Z6.3+sync+po+addr } &  &  & \cmark & \cmark & \cmark \\
\texttt{Z6.3+sync+po+po } &  &  & \cmark & \cmark & \cmark \\
\texttt{Z6.3+sync+po+sync } &  &  & \cmark & \cmark & \cmark \\
\texttt{Z6.3+syncs } &  &  &  &  &  \\
\texttt{Z6.3+sync+sync+addr } &  &  &  &  &  \\
\texttt{Z6.3+sync+sync+po } &  &  &  & \cmark & \cmark \\
\texttt{Z6.4 } &  & \cmark & \cmark & \cmark & \cmark \\
\texttt{Z6.4+po+po+sync } &  & \cmark & \cmark & \cmark & \cmark \\
\texttt{Z6.4+po+sync+po } &  & \cmark & \cmark & \cmark & \cmark \\
\texttt{Z6.4+po+sync+sync } &  &  & \cmark & \cmark & \cmark \\
\texttt{Z6.4+sync+po+po } &  & \cmark & \cmark & \cmark & \cmark \\
\texttt{Z6.4+sync+po+sync } &  & \cmark & \cmark & \cmark & \cmark \\
\texttt{Z6.4+syncs } &  &  &  &  &  \\
\texttt{Z6.4+sync+sync+po } &  & \cmark & \cmark & \cmark & \cmark \\
\texttt{Z6.5 } &  & \cmark & \cmark & \cmark & \cmark \\
\texttt{Z6.5+po+po+sync } &  &  & \cmark & \cmark & \cmark \\
\texttt{Z6.5+po+sync+po } &  & \cmark & \cmark & \cmark & \cmark \\
\texttt{Z6.5+po+sync+sync } &  &  & \cmark & \cmark & \cmark \\
\texttt{Z6.5+sync+po+po } &  & \cmark & \cmark & \cmark & \cmark \\
\texttt{Z6.5+sync+po+sync } &  &  & \cmark & \cmark & \cmark \\
\texttt{Z6.5+syncs } &  &  &  &  &  \\
\texttt{Z6.5+sync+sync+po } &  & \cmark & \cmark & \cmark & \cmark \\
\\
Count & 0 & 35 & 89 & 140 & 155
\end{longtable}



\end{document}
